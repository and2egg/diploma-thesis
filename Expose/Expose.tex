\documentclass[a4paper]{article}

\usepackage[english]{babel}
\usepackage[utf8]{inputenc}
\usepackage{csquotes}
\usepackage{amsmath}
\usepackage{graphicx}

% Load in biblatex
% To use a different bibliography style, just change "numeric" to
% your preferred style (mla for MLA style, alphabetic for Author-Year
% style, etc.) There are a lot of options; check the BibLaTeX documentation.
\usepackage[backend=bibtex8,style=numeric]{biblatex}
% Select the bibliography file
\addbibresource{sources.bib}


\title{Cost aware management of virtual machines in distributed data centers}
\author{Andreas Egger \\0626885}
\date{Technical University of Vienna}

\begin{document}
\maketitle


\section{Problem Statement}

In times of cloud computing where computer resources are expected to be readily available from anywhere \cite{buyya2009cloud} and with an ever growing global network of interconnected devices energy efficiency plays an increasingly important role in delivering and consuming cloud based services. 

As stated in \cite{qureshi2009cutting} internet scale systems consume a high amount of power annually where the electricity bill for such systems is in the range of several millions of dollars per year. Due to increasing energy consumption in data centers and high energy prices advancements in cloud computing may be slowed down or hindered. Thus, cloud providers must ensure that delivering their services remains profitable. Lower energy costs increase the competitive ability of cloud providers and bigger investments are possible. Likewise customers may benefit from more affordable cloud services. 


The goal of this thesis is to reduce energy costs of interconnected data centers while maintaining defined quality of service levels. The proposed approach is to optimize load migration between data centers considering time and location based energy prices. The scenario includes data centers located in different countries where different energy markets and prices apply. Thus constellations may arise where it is profitable to migrate a certain amount of resources from a data center facing high energy prices to a data center where energy prices are low at the current point in time or in the near future. 

Energy prices are traded and change hourly within the energy markets, so both long and short term investigations need to be made in order to make qualified statements about the estimated energy price in a specific area.
Investigations are done by analyzing historical price data and building models for accurate energy price forecasting. Energy prices belonging to different energy markets are estimated for various time ranges in the future, i.e.~one day up to several days. Based upon the results it is decided how many resources may be migrated for how many days in order to obtain the maximum possible energy cost reduction. However, there is a gap between estimated and real energy prices which is represented by forecasting error measures (i.e.~MAPE). The challenge is that in general the error increases with the length of the estimation window which has to be considered when investigating the results. 


%One way to lower the operating expense is to increase energy efficiency in data centers. This can be done by changing the infrastructure, i.e.\ by investing in more energy efficient hardware, or by applying energy aware scheduling algorithms for managing virtual machines. Another way is to build data centers in locations where energy prices are low or the cooling infrastructure is supported by the outside temperature. Energy prices are traded and change hourly within the energy markets, so long term investigations need to be made in order to make qualified statements about the estimated energy price in a specific area.

Since energy prices change dynamically a flexible approach of moving data center resources to more cost-effective locations is needed. This work is based on the scenario of a company running data centers distributed across several countries. Depending on the current and estimated energy price for a certain region resources may be relocated to other data centers located at more cost-effective regions.

The key technology to implement the relocation of data center resources is virtual machine (VM) migration \cite{nelson2009virtual}. All data and services are located on servers and belong to specific virtual machines within a data center. Since a server may host several VMs that operate completely independent from each other, they can be relocated (migrated) to other servers, possibly over networks without affecting VMs running on the same server. VM migration may be applied across large geographical distances (geo migrations) by leveraging connections via the internet. It is also possible to do live migrations on a large scale without noticeable service interruption \cite{celesti2010improving}. 

Various load policies exist to optimize the load factor and distribution of tasks within data centers \cite{buyya2010energy}. A workload manager has to make sure that there are still servers having enough resources and no service level agreement (SLA) is violated on the assignment of a new job. 
This work additionally aims to optimize workload distribution among several data centers with respect to changing energy prices. 
%For an optimal distribution of tasks the current load factor of a data center needs to be taken into account. 
%A job definition comprises parameters like the maximum required CPU power, RAM and disk space. 
Furthermore, by considering the history of energy prices and resource load within a certain region or data center it is estimated whether or not a migration to another data center makes sense regarding the current load and configuration. Therefore, forecasting methods in combination with machine learning techniques will be evaluated and applied to the scenario. 

%In real world scenarios additional parameters need to be considered such as the total availability of CPU power, available RAM and disk space. For this work the scenario is simplified to only consider CPU power, which is the only parameter that directly affects energy consumption and thus energy costs as well. In addition, by considering the history of energy prices and resource load within a certain region or data center the most promising date of migration may be estimated and whether a migration makes sense regarding the current load and configuration. Therefore, forecasting methods in combination with machine learning techniques will be evaluated and applied to the scenario. 



\section{Expected Results}

The aim of this thesis is to provide insights on how to cost-efficiently manage geographically distributed data centers within a cloud. For this reason a simulation framework will be established that illustrates the optimal placement of data center resources considering changing energy prices and workload for a given time frame. 

The simulation will be based on a predefined scenario of a company running its own cloud consisting of data centers located in different countries. The electricity price will be estimated for various time intervals in the future. Since the prices of a day ahead market are determined simultaneously for each successive 24 hours of a single day the time difference between two forecasts is a full day or multiples thereof. The predicted energy prices are used to decide if short term or long term migrations would lead to substantial cost savings while taking into account resulting migration overheads. 
By applying forecasting methods on workload and energy price history it can be determined if a VM migration may be justified in the near future or not at all. 
The results should outline which duration of resource migration result in a maximum of cost savings on average given certain preconditions. 
%This should help cloud providers make informed decisions to save energy costs by letting virtual machine migrations be initiated between data centers at the right time. 

The framework should be able to outline which energy savings are possible considering various preconditions. 
These preconditions comprise different load and migration policies where virtual machines may be migrated to other servers within the data center to optimize the distribution of tasks. This allows to use fewer servers to host the VMs and to switch off the remaining idle ones. Thus, the overall energy consumption and energy costs may be reduced. 
A visual representation should provide both an overview of parameters of various clouds and detailed behavior of workload and price traces. 

Current and historical energy prices will be obtained from wholesale energy markets valid at different data center locations. The framework will comprise seven data centers located in different countries and assigned to four different energy markets, namely the 
European energy exchange (EEX)\footnote{EEX: European energy exchange, http://www.eex.com/}, Nord Pool Spot\footnote{Nord Pool Spot, http://www.nordpoolspot.com/}, APX Power Spot Exchange\footnote{APX Power Spot Exchange, http://www.apxgroup.com/} and ISO New England\footnote{ISO New England, http://www.iso-ne.com/}. Thus, the simulation framework can make realistic assumptions of when and how to migrate virtual machines in distributed data centers and therefore can be a valuable contribution to current research using real data from wholesale energy markets. 


\section{Methodological Approach}

The first step will be to define the scenario with a private cloud comprising data centers located in countries within the proposed energy markets. The geographical location of the data centers should be defined and connections should be established such that VM migrations may be done interchangeably. 

Then a model should be defined with parameters needed for further operation. For example, the amount of bandwidth between two data centers needs to be set and a method for calculating the cost of VM migration has to be defined, depending on current energy prices. Furthermore the minimum block size (number of VMs) should be defined that may be migrated in order to compensate for the migration costs.

As a next step test data for work load should be generated (i.e.\ CPU Power load) and distributed to the data centers. This should reflect real operation and support the testing framework. Actual and historical energy price data is available from the wholesale energy markets and is valid for certain countries, respectively. 

%Different forecasting methods should be evaluated and compared to find the most accurate one regarding this scenario. The accuracy should be evaluated for various time ranges, i.e.\ one day up to a few months. The models should first be trained with generated test data consisting of different mathematical functions to estimate trends and show the validity of the models. The models are then applied to the workload and energy price data depending on the chosen time interval. 

To maximize the accuracy of data forecasting different parameters of forecasting methods should be evaluated and compared to get the most accurate results regarding this scenario. The accuracy should be evaluated for various time ranges, i.e.\ one day to a few days. Therefore the models should be trained on historical data and tested on out of sample data to ensure their validity. 

Since intraday prices in energy markets are published simultaneously for each day there is no direct correlation between single hourly prices. Therefore, to obtain accurate results, multivariate forecasting models should be used that provide forecasting series for each hour separately. Care has to be taken to restrict the complexity of the models to prevent overfitting and provide accuracy. 

A vector autoregressive (VAR) model will be used, specifically a Bayesian VAR model, to cover these requirements \cite{raviv2013forecasting}. 
In addition, it is well known that a combination of forecasting models can provide better results than each of the individual models \cite{raviv2013forecasting}. Therefore a combination of time series models, neural networks and regression trees will be used to make accurate predictions for short and mid-term forecasting. 

The simulation framework should be built as a web application, having a visual representation of the interconnected clouds with the option to have a detailed look at a cloud’s workload and energy price behavior in that region. The time interval may be adjusted dynamically to get insights into the data flow at any time.

There should be a front end managing the data flow of all data centers. The data flow is represented as a workload trace containing job specifications with a defined start and end time, maximum CPU load and number of required CPU cores. According to these specifications the jobs are distributed to a data center depending on current load and energy prices. 
%Since a virtual machine represents a job running on one CPU multi core jobs may be distributed across several VMs and servers. 
Due to changing energy prices and workload it might be advantageous or necessary to migrate jobs to another data center. It might also be necessary to delay jobs such that cheaper energy prices are exploited. However, jobs may not be delayed beyond a defined threshold in order to ensure compliance of SLAs. 

%Applications with predefined workload (high performance or web applications) may be started and stopped at any data center, possibly creating the need for migration. In addition certain constraints may be defined to prevent disparate workload distributions and data center overload. A time frame can be chosen for which energy and price estimations are calculated. At any given point in time suggestions are displayed to the user as to how many virtual machines currently need to be migrated in order to meet requirements or to reduce energy costs. The user then should decide which action to take. There should be the possibility of prioritizing energy costs over optimal workload distribution. 


\section{State-of-the-art}

Concerning electricity costs a lot of ongoing research is in progress \cite{guler2013cutting, le2011reducing}. A comparison of different approaches considering spatial and temporal energy price changes and the contribution of the outside temperature to the resulting cooling effort and expense is outlined in \cite{guler2013cutting}. In \cite{le2011reducing} a set of homogenous data centers is simulated where the focus lies on the effects of changing workloads and migrations on the cooling infrastructure and on intelligent scheduling to reduce energy consumption and energy costs, respectively. 

%In this scenario a network of geographically distributed data centers is built where not only energy costs but also penalties of SLAs are taken into account. For different workloads and scheduling algorithms one can derive an optimal distribution of tasks which can lead to effective cost savings. 

The authors in \cite{lucanin2013take} propose a green cloud approach using real time electricity prices where the duration and time of price peaks are estimated and VMs are paused during these times to reduce energy costs and save energy. Customers may choose between ``green'' and normal instances, taking into account some loss of availability for a reduced price. 

Time series analysis is an active research topic as it is important to find efficient algorithms that scale on very large amounts of data. A survey on machine learning models in time series forecasting is presented in \cite{ahmed2010empirical}, where the suitability of the models concerning forecasting is evaluated. Notably, some models did considerably better than others, therefore these models are good candidates for applications in time series forecasting. 

In \cite{lin2011pattern} a comprehensive benchmark on multivariate time series is described using similarity and distance measurement functions and pattern recognition techniques. In this work unsupervised / semi-supervised machine learning methods are investigated and a new similarity measure is introduced which exhibits a very high hit rate. 

There are different proposed approaches related to virtual machine migration in distributed cloud environments \cite{celesti2010improving, malet2010resource}. In \cite{celesti2010improving} the so-called Composed Image Cloning (CIC) methodology designed for migration of VMs across federated clouds is introduced. This work aims to reduce the needed bandwidth and migration time by setting up a new virtual machine in the destination cloud and transferring only user data instead of relocating the whole VM disc image. 

In another work \cite{malet2010resource} the placement of applications is dynamically adjusted across distributed data centers according to the location of the currently highest request rate. Agarwal et al.\ \cite{agarwal2010volley} provide a framework to minimize inter-data center traffic and user perceived latency by analyzing log files and thus determine the best placement and migration strategy. 


\section{Relation to the Study of Software Engineering}

This work aims to provide a contribution to cloud based scheduling applications focusing on reducing the overall energy expenses of interconnected data centers. Due to the growing popularity of cloud and internet based services these topics have found their way into the curriculum of ``Software Engineering \& Internet Computing'' \cite{curriculum2013curriculum}. 

The following constitutes a list of the most relevant courses related to this topic:


\begin{itemize}

\item{184.269 VU Advanced Internet Computing}
\item{184.260 VU Distributed Systems Technologies}
\item{184.738 VU Computer Networks}
\item{184.153 VU Distributed Systems Engineering}
\item{184.271 VU Large-scale Distributed Computing}
\item{183.286 VU Mobile Network Services and Applications}
\item{184.706 VO Network Engineering}
\item{184.707 UE Network Engineering}
\item{184.194 SE Seminar in Distributed Systems}
\item{184.163 VU Service Level Agreements }

\end{itemize}

\printbibliography
\end{document}