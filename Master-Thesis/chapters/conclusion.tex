




ARIMA models showed reasonably good performance, in many cases they showed the best results compared to other evaluated models in this work. 







In \cite{de2013study} it is shown that the cost of optimal assignment is further reduced by increasing the number of locations with data centers assigned to different energy markets and/or by increasing the variability of energy prices. 


\section{Benefits and limitations}



...

As Qureshi et al.~pointed out in \cite{qureshi2009cutting} most companies would need to renegotiate their energy contracts in order to utilize the proposed approach. Companies renting space in co-location facilities usually pay a fixed price for energy rather than directly taking part in the price offers of wholesale markets. 

Demand response programs are a way of further decreasing energy costs as consumers agree on a reduced price in exchange for a reduced load when it is requested by the grid operator\cite{albadi2008summary}. This is certainly interesting for energy elastic distributed systems that are able to shift load to other locations dynamically. 






\section{Future Work}


Forecasting models may be improved, or different models may be investigated based on electricity price data. 

Dynamic Regression (DR) and Transfer function (TF) models have been shown to provide a significantly better performance than ARIMA models \cite{aggarwal2009electricity,weron2005forecasting}. This may be due to the additional regressor variables that are considered in multivariate models (DR and TF) in contrast to models based on a single univariate time series (ARIMA). 

In general multivariate models seem to provide better results than univariate models as they are able to include more information into the model generation process \cite{weron2005forecasting}. 
