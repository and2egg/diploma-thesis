\chapter*{Kurzfassung}

Cloud Dienste haben in den letzten Jahren zunehmend an Popularität gewonnen wodurch die Energiekosten von Datenzentren signifikant stiegen. 
Da die Energiekosten einen bedeutenden Anteil an den Gesamtkosten eines Datenzentrums betragen ist es für Cloud Betreiber von großer Bedeutung diese auf ein Minimum zu reduzieren um wettbewerbsfähig zu bleiben und leistbare Cloud Dienste anbieten zu können. 
Diese Arbeit präsentiert ein Cloud Framework zur Evaluierung von Energiekosteneinsparungen mit Anbindung an unterschiedliche Elektrizitätsmärkte. Es wird gezeigt dass signifikante Kosteneinsparungen mit Hilfe von intelligenten Verteilungsmechanismen unter Berücksichtigung von Energiepreisvorhersagen möglich sind. 


Die Arbeit besteht aus zwei Teilen. Im ersten Teil werden unterschiedliche Vorhersagemodelle untersucht um die Qualität der Modelle bezüglich Energiepreisdaten unterschiedlicher Energiemärkte zu bestimmen. Ein Simulationsframework für Vorhersagemodelle wurde erstellt das die dynamische Evaluierung von Energiepreisdaten ermöglicht und in der Lage ist Vorhersagemodelle basierend auf diesen Daten zu erstellen. 
Das Framework stellt Methoden zur automatischen Modellgenerierung bereit das als Basis für weitere Simulationen dient. 
Zudem wird eine weitreichende Auswertung von Vorhersagen durchgeführt um Erkenntnisse über die Genauigkeit der Vorhersagen für eine größere Menge an Energiepreisdaten zu erhalten. Verschiedene Trainingsperioden, Vorhersagezeiträume und Auswahl an Energiedaten ergeben eine umfassende Analyse der Vorhersagemodelle. 


Im zweiten Teil werden Simulationen über größere Zeiträume hinweg durchgeführt mit unterschiedlichen Verteilungsalgorithmen und Szenarien. Ein bestehendes Simulationsframework wurde erweitert um Simulationen über größere Datenmengen von Energiepreisen durchzuführen wobei verschiedene Verteilungsmechanismen vorgestellt werden. 
Eine Nutzenfunktion wurde definiert die basierend auf unterschiedlichen Kriterien eine Entscheidungshilfe bereitstellt um die best geeignetsten virtuellen Maschinen zur Migration zu einem bestimmten Zeitpunkt zu ermitteln. 
Beispiele von Kriterien sind die Wahrscheinlichkeit von SLA Strafen und die maximal möglichen Kostenreduktionen die entsprechend gewichtet werden um einen Kompromiss zwischen der Kostenreduktion und den dadurch entstehenden SLA Strafen zu erreichen. 


Resultate zeigen dass erhebliche Kosteneinsparungen möglich sind unter Verwendung von Verteilungsmechanismen die unterschiedliche Kriterien und Energiepreisvorhersagen berücksichtigen. Basierend auf der simulierten Cloud Umgebung kann die Anwendbarkeit der vorgestellten Methode auf reale Umgebungen bestehend aus geographisch verteilten Datenzentren mit Anbindung an Energiemärkte gezeigt werden. 



