\chapter*{Abstract}

Cloud computing services experienced increasing popularity over the recent years which caused the power demand of data centers to increase significantly. 
As energy costs represent a significant part of the total cost of data centers energy cost reductions in cloud computing environments play an important role for cloud providers to remain competitive and provide affordable cloud services. This thesis introduces a cloud framework to evaluate energy cost reductions in multi electricity market environments. It is shown that substantial cost savings are possible using intelligent resource scheduling algorithms with integration of energy price forecasts. 

This thesis consists of two parts. In the first part different forecasting models are evaluated to assess the performance of the models for energy price time series of different energy markets. A forecast simulation framework is introduced capable of dynamically managing energy price data from different power markets and generating forecasts based on that data. The framework incorporates methods for automatic model generation and evaluation as a basis for extended forecast simulations. 
A large scale forecast evaluation is conducted to gain insights into model accuracy across a wide range of energy price data. Different training periods, forecast horizons and energy price datasets lead to a comprehensive evaluation of the forecasting models. 

The second part is dedicated to large scale cloud simulations comprising different cloud schedulers and scenarios. An existing cloud simulation framework has been extended to perform simulations across a wide range of energy price data where different scheduling mechanisms have been introduced. 
A utility function has been defined for sophisticated evaluation of different criteria to decide on the best suitable VMs to migrate at any point in time. Criteria involve probability of SLA penalties and maximum cost benefits to appropriately handle the tradeoff between minimizing costs and providing an adequate level of quality of service. 

Results show that significant cost savings are possible with schedulers incorporating different criteria and energy price forecasts revealing promising results. Based on the defined cloud settings it illustrates the applicability of the proposed approach to real world scenarios of geo-distributed data centers connected to energy markets. 


