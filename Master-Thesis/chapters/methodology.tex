

\section{Cloud Scenario}

%\subsection{Assumptions}
%
%\begin{itemize}
%
%\item Energy Elasticity
%
%One of the key assumptions for the cloud scenario is a reasonable level of server energy elasticity. It is defined as the difference between power values at idle and peak state regarding CPU utilization. 
%
%\end{itemize}


\section{Cloud Simulation}

\subsection{Benefits and limitations}

\subsection{Simulation assumptions}

\subsection{Expected results}




\section{Cost models}

%TODO reference section
As discussed before virtual machine migration is vital to draw benefits from the proposed approach. The targeted benefits of this approach are overall cost reductions in a geographically dispersed cloud environment. 

As energy prices are provided in e.g.~MegaWatt Hours (MWh) the amount of consumed energy per time is needed to calculate resulting costs. Typically several factors contribute to the overall energy consumption in a data center such as server power consumption or cooling infrastructure. One efficiency measure of data centers is the Power Usage Effectiveness (PUE) that indicates the power needed for auxiliary devices vs.~the power drawn solely from IT equipment (e.g.~servers or network hubs). 

Even though cooling expenses play a significant part in the overall energy costs this work only considers servers' power consumption as it is most significant to the proposed scenario. 

A server's power consumption consists of a static and a dynamic part \cite{liu2013performance}. The static part is defined as the power drawn when a server is idle i.e.~it does not run any computations. The dynamic part varies according to the current data processing on the server which can be derived from the current utilization of hardware components. 


\subsection{Cost of optimal assignments}


In \cite{de2013study} it is shown that the cost of optimal assignment is further reduced by increasing the number of locations with data centers assigned to different energy markets and/or by increasing the variability of energy prices. 




%\subsection{Forecasting in cloud environments}


\section{SLA management}

As mentioned before the proposed approach relies on Virtual Machine migrations to shift load to data centers with currently cheaper energy prices. Each VM migration causes a temporary downtime depending on network bandwidth and VM memory \cite{liu2013performance} which has to be considered when applying this approach. 

Big cloud providers such as Google or Amazon commonly offer high availability services with a guaranteed availability of 99\% or even 99.9\%\footnote{Google Service Level Agreements \url{https://cloud.google.com/storage/sla}}\footnote{Amazon Service Level Agreements \url{https://aws.amazon.com/de/ec2/sla/}}. However when applying geotemporal VM migrations within a cloud environment it is not possible to keep availability at these levels due to VM downtimes. Therefore an alternative approach considering flexible SLA offers needs to be applied to accommodate possible downtimes \cite{luvcanin2014energy}. 


%\section{Virtual Machine migration}



%\subsection{Application Server}
%
%Considering the need for easy and generic data handling when dealing with different energy markets and data formats the application server has been set up which is connected to a database to fetch and parse data for later processing. By this means it is possible to execute extended data analysis tasks and aggregate and combine data to get insights into possible relationships between different data sets. Scenarios based on different datasets are easily created to quickly run simulations under changing conditions. Thus it is a flexible way of data handling that also simplifies the whole simulation process. 
%
%Another benefit of providing an application server is the possibility to expose web service interfaces that can be called by any external component to utilize data and forecast interfaces. The architecture has been made extensible such that a new energy market location can be added without touching existing interfaces. 
%Data processing interfaces exist for different data formats which can be extended to provide implementations for different energy price data formats. 
%
%\subsubsection{Data management}
%
%Interfaces for data management are defined within specialized Java classes that are responsible for the retrieval, import and data query handling. 
%
%\subsubsection{Web Services}
%
%All publicly exposed interfaces are designed as web services that can be called by any application regardless of its type or implementation. Specifically REST service interfaces have been chosen as it is a simple yet powerful method of providing reliable interfaces for interactions with the server. The underlying Java technology is JAX-RS which is a common implementation for providing REST service interfaces on Java application servers. Data may be passed by Path or Query parameters that are specified at each service interface. 
%
%\subsubsection{Scheduling}
%
%The application server also includes scheduling services where actions are triggered at each interval. This relates to the continuous and automatic import of data from various energy markets and the subsequent generation of R models. Therefore at each service invocation the respective REST interfaces are called to import the latest data from the markets and trigger the model generation based on that data. This mechanism allows for statistical models to be always up to date and are saved and retrieved based on the last date of the underlying training data. 
%
%
%
%\subsection{R Server}
%
%R is a statistical tool with vast amounts of statistical methods for data analysis and model generation\footnote{\url{https://www.r-project.org/}}. It has been chosen as the statistics engine for this work since it does not require a commercial license and became the de-facto standard for statistical processing over the last decade. 
%
%
%



%[TODO - move this paragraph to Simulation]

%In this thesis different energy markets have been chosen that exhibit a considerable degree of price volatility and are partly located in different timezones. Therefore different scenarios may show interesting characteristics due to time- and location based differences in energy prices. 

%A cost function is provided which calculates the total energy cost based on the energy price, the power used by a single server and the total number of running servers for each location. In addition constraints are defined s.t.~the workload assigned to a DC does not exceed its capacity and the total number of requests does not exceed a predefined threshold. 



%
%\section{Data center characteristics}
%
%\subsection{Key indicators in data centers}
%
%\subsection{Data management}
%
%\subsection{Best practices in federated cloud environments}




\section{Summary}



