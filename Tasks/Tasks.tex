\documentclass[a4paper]{article}

\usepackage[english]{babel}
\usepackage[utf8]{inputenc}
\usepackage{csquotes}
\usepackage{amsmath}
\usepackage{graphicx}

% Load in biblatex
% To use a different bibliography style, just change "numeric" to
% your preferred style (mla for MLA style, alphabetic for Author-Year
% style, etc.) There are a lot of options; check the BibLaTeX documentation.
\usepackage[backend=bibtex8,style=numeric]{biblatex}
%% Select the bibliography file
%\addbibresource{sources.bib}


\title{Diploma Thesis\\Tasks}
\author{Andreas Egger \\0626885}
\date{Technical University of Vienna}


\begin{document}
\maketitle

\newpage

\tableofcontents

\newpage

\hfill\date{Week 15, from 07.04. to 13.04.}

\section{Defining the simulation}

\subsection{Defining simulation conditions}

\begin{itemize}

\item Defining countries and locations of data centers

\item Creating a visual outline of the geographical structure

\item Defining parameters of the simulation

\item Collecting energy price data from the various energy markets

\item Examining data formats and ways to process them

\end{itemize}

\vspace{1em}

\hfill\date{Week 16, from 14.04. to 20.04.}

\section{Forecasting}

\subsection{Experiment on forecasting methods}

\begin{itemize}

\item Examine the forecasting methods in relation to the actual energy price data

\item Finding metrics for accuracy of the methods

\item Setting up a test environment for testing forecasts

\item Test forecasting for different time ranges (day ahead, few days)

\item Compare the accuracy of forecasts for different models

\end{itemize}


\vspace{1em}

\hfill\date{Week 17, from 21.04. to 27.04.}

\section{Energy Data and Forecasting}

\subsection{Experiment on forecasting methods (Re-work last week's tasks)}

\begin{itemize}

\item Examine the forecasting methods in relation to the actual energy price data

\item Finding metrics for accuracy of the methods

\item Setting up a test environment for testing forecasts

\item Find a way to feed energy price data to Matlab forecasts

\item Test forecasting for different time ranges (day ahead, few days)

\end{itemize}


\end{document}