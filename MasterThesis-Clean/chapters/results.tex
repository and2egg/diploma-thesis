

\section{Architectural outline}

\subsection{Application Server}

\subsubsection{Datamanagement}

\subsubsection{Web Services}

\subsubsection{Scheduling}


\subsection{Simulator and Scheduler}

\subsubsection{Basic structure}

\subsubsection{Configuration}

\subsubsection{External interfaces}

\subsubsection{Scheduler}


\subsection{R Server}

\subsubsection{Forecast Models}

\subsubsection{Assessment of Forecasts}

\subsubsection{Forecast error evaluation}



% Scheduler Types

% Benefits and drawbacks
 
% Settings and configuration



%\section{Cloud-based Simulator}

%\subsection{Motivation}
%
%\subsection{Structure of simulator}
%
%\subsection{Functional requirements}
%
%%TODO 
%Discuss impact of evaluation of bandwidth costs. In \cite{rao2010minimizing} %(page 7) 
%bandwidth costs are deliberately neglected, but the option should be still taken into account. As these costs directly relate to costs of migrations it is vital to include and model them within the scenarios. 
%
%\subsection{Non-functional requirements}
%
%\subsection{Incorporating forecast models}
%
%As already discussed in the introduction %TODO WHERE ?(forecast chapter? TODO) 
%forecasting is an integral component of the simulation as it should support in decision making when assigning workload and performing migrations. Since forecast errors can have a negative impact on workload allocation which may result in non-optimal workload distributions \cite{de2013study} it is of paramount importance that the forecast models are trained thoroughly and forecast errors are kept at a minimum. In order to reduce the impact of forecast errors in the simulation forecasting is done several hours into the future whereby the results are averaged to obtain a broader estimate of energy prices of the near future. 



\section{Model selection}

\subsection{Model type selection}

\subsection{Model selection based on data}

\subsection{Accuracy of selected models}


\section{Simulation and Scheduling}

\subsection{Cloud settings}

\subsection{Cost models}


Cost models are required to map the servers' energy consumption to costs depending on current energy prices. Different cost models exist exhibiting higher or lower accuracy and differ in their type of calculation. 

In this thesis several assumptions have been made: 

\begin{itemize}

\item \textbf{idle power.} A server has a defined idle power (e.g.~100 Watts (W)). This is the power that is drawn when the server is in idle mode (no tasks are running). 

\item \textbf{peak power.} A server has a defined peak power (e.g.~200 Watts (W)). This is the power that is drawn when the server runs on maximum load (maximum number of tasks running). 

\item \textbf{utilization.} Each server has a utilization between 0 and 100 percent. A utilization of zero percent means the server is idle, in contrast a utilization of hundred percent means the server is at peak load. 

\item \textbf{power consumption.} The power consumption is defined as the effective power consumed by a server during a certain period of time (e.g.~one hour). The result is a measure of power consumed per time (e.g.~150 Watt hours (Wh) for a server running on 50\% utilization for one hour)\footnote{\url{http://whatis.techtarget.com/definition/kilowatt-hour-kWh}}. %Energy can also be measured in Joule (J) where one Watt (W) equals 1 Joule per second (J/s)\footnote{\url{http://en.wikipedia.org/wiki/Kilowatt_hour}}. Thus one kilowatt hour equals $3.6 * 10^6$ kilojoules. 

\item \textbf{energy price.} The energy price is taken from the energy market that the data center is connected to. These prices typically change every hour which is therefore the relevant time interval for all further operations. 

\end{itemize}

From these assumptions a simple cost model can be derived. Since power is measured per hour in Watts and prices are given per Kilowatt Hours (kWh) the power consumed by servers can be directly mapped to energy costs. This model is simplified in that it does not consider frequency scaling or other power measures taken for a server (suspend, sleep mode). Thus it serves as a base for comparing different scenarios and it is also useful to get a first impression on the overall power consumption and costs involved in a given scenario. 

A cost model is essential for evaluating the usefulness of the approach presented in this thesis as the ultimate goal is to reduce energy related costs of geographically distributed and connected data centers based on changing energy prices. 



\subsection{Cloud Scheduler}

\subsection{Simulation scenarios}

\subsection{Simulation results}


\section{Statistics and Empirical Evaluation}

\subsection{Result evaluation}

\subsection{Possible improvements}

\subsection{Relevance of results}



