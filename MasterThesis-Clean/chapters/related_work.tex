

\section{Literature studies}

\subsection{Electricity markets and pricing}

In wholesale energy markets different pricing and bidding models can be used. Currently the two most common price evaluation strategies are day-ahead and real-time pricing strategies. 

\subsubsection{Bidding strategies}

In \cite{tierney2008uniform} two different bidding strategies are discussed, uniform pricing and pay-as-bid auctions. In the uniform pricing model the market clearing price is determined by collecting the marginal prices from all suppliers and taking the maximum price from this collection. Conversely, in pay-as-bid auctions a supplier gets paid based on its actual bid. 
The second approach may seem beneficial from the customer's point of view since suppliers may set individual prices which enables competition within the market. 
However studies show that in this pricing scheme suppliers set their prices at the maximum possible level to be comparable to other suppliers and keep their customers. On the other hand the uniform pricing model provides a uniform clearing price which is valid for all participants in the market and customers may trust that suppliers set prices to just satisfy their needs. 

\subsubsection{Cost optimization}

In \cite{rao2010minimizing} a cloud consisting of several Internet Data Centers (IDC) is examined with regard to cost optimization which is done by intelligently assigning workload to data centers based on current energy prices. Two of these data centers in the scenario are connected to deregulated wholesale energy markets whereas the other one is connected to a regulated utility region charged by a fixed pricing scheme. Since price change differences in deregulated energy markets can be considerable energy cost reductions may be achieved by assigning workload based on changing energy prices.  %For the purpose of the simulation five front-end Web portal servers process a total workload of about $10^5$ requests per second. 

In this paper the approaches of optimal and average workload assignments are visually and formally compared to gain insights into the total electricity cost reduction. The optimal workload assignment is calculated with respect to workload and delay constraints that are considered in a minimum cost flow problem based on a linear programming model. Thus the goal is to minimize costs without compromising quality of service constraints. 
%The proposed cost reductions amount up to 30 percent for a single hour test within the given time range. 
%The results measured at two timestamps at a specific day seem promising, since total cost reductions could be achieved by 30\% and 17\% respectively. 

The differences compared to this work is that in \cite{rao2010minimizing} they do not consider longer running jobs as used in scientific calculations or large optimization problems that may take several hours to complete. Thus the impact of migrations is not examined in this scenario. In addition the possibility of energy price forecasting is not considered which may result in even greater energy cost reductions. 

Another study on cost reductions in Internet Data Centers has been done where different aspects of cost reductions including cost prediction schemes have been considered \cite{de2013study}. Since up to 15\% of total capital investment is spend on energy related costs 
%(paper references \cite{greenberg2008cost} - old, from 2009!!) 
a special effort is invested to reduce the amount of energy costs through cost aware operations. %(TODO: Move to introduction?)

In this paper cost optimization is seen as an assignment problem where an overall cost function is minimized through intelligent workload allocation. During the investigation different variables and their impact on resulting energy costs are considered which are price volatility, price predictions including variable prediction errors, time lag between locations and reconfiguration delays. 

Similar to this work the simulation is run with the same set of fixed parameters to evaluate the impact of the various scenarios. It is observed that prediction errors greatly impact workload allocation where cost penalties due to non-optimal assignments increase quadratically with an increase of forecast errors \cite{de2013study}. Also with increasing number of locations the minimum cost of optimal assignments can be greatly reduced. Greater price volatility is beneficial as well since then price aware assignments will have greater impacts. 

A comprehensive study on cost reductions and energy market characteristics in an environment where data centers are placed within the reach of different energy markets is presented at \cite{qureshi2009cutting}. Based on geotemporal variations of energy prices the maximum cost reductions under different scenarios are evaluated. Energy expenses are estimated for various large scale companies like Google and Yahoo to state the actual savings and the amount of reductions in energy costs that would be possible. 
They also discuss the impact of considering bandwith constraints and maximum client server distances on cost reductions. 

Interestingly enough long term seasonalities in the energy price data has been discovered that spanned multiple energy markets. This is an important fact when training forecasting models since predictions can become a lot more accurate. An important fact that was revealed about energy markets was that data from different energy markets was much less correlated than data from the same area. Thus to take full advantage of energy price differences data centers located at different energy markets should be combined \cite{qureshi2009cutting}. 

With different state of the art energy models and simulation constraints a summary of possible cost reduction schemes was presented. However no migration or forecasting of energy price data has been done which could further improve scenarios with longer running jobs. In addition these cost reductions are only valid under specific assumptions that large cloud providers would need to implement such as connection to wholesale energy markets and reasonable server energy elasticity. 



%[TODO - move this paragraph to Introduction]
In this thesis different energy markets have been chosen that exhibit a considerable degree of price volatility and are partly located in different timezones. Therefore different scenarios may show interesting characteristics due to time- and location based differences in energy prices. 


%A cost function is provided which calculates the total energy cost based on the energy price, the power used by a single server and the total number of running servers for each location. In addition constraints are defined s.t.~the workload assigned to a DC does not exceed its capacity and the total number of requests does not exceed a predefined threshold. 

\subsection{Energy cost reduction in data centers}

Active research topics include electricity cost reduction in data centers \cite{guler2013cutting, le2011reducing}. A comparison of different approaches considering spatial and temporal energy price changes and the contribution of the outside temperature to the resulting cooling effort and expense is outlined in \cite{guler2013cutting}. In \cite{le2011reducing} a set of homogenous data centers is simulated where the focus lies on the effects of changing workloads and migrations on the cooling infrastructure and on intelligent scheduling to reduce energy consumption and energy costs, respectively. 

The authors in \cite{lucanin2013take} propose a green cloud approach using real time electricity prices where the duration and time of price peaks are estimated and VMs are paused during these times to reduce energy costs and save energy. Customers may choose between ``green'' and normal instances, taking into account some loss of availability for a reduced price. 


\subsection{Power management}

A well known problem in power management is how to accurately and efficiently model a server's power consumption over time. At \cite{horvath2008multi} Horvath et al.~exploit dynamic voltage scaling (DVS) and multiple sleep states to reduce power consumption of a server cluster of about 23\% without significantly impacting performance. They propose that CPU utilization and frequency are the variables that have the most significant impact on the power consumption of a machine. This assumption is also used in other studies, e.g.~\cite{rao2010minimizing, hammadi2014survey, kliazovich2012greencloud}. 

\subsection{Time series forecasting}

Time series analysis is an active research topic as it is important to find efficient algorithms that scale on very large amounts of data. A survey on machine learning models in time series forecasting is presented in \cite{ahmed2010empirical}, where the suitability of the models concerning forecasting is evaluated. Notably, some models did considerably better than others, therefore these models are good candidates for applications in time series forecasting. 

In \cite{lin2012pattern} a comprehensive benchmark on multivariate time series is described using similarity and distance measurement functions and pattern recognition techniques. In this work unsupervised / semi-supervised machine learning methods are investigated and a new similarity measure is introduced which exhibits a very high hit rate. 

\subsection{Machine Learning approaches in forecasting}

\subsection{Virtual machine migration in cloud environments}

There are different proposed approaches related to virtual machine migration in distributed cloud environments \cite{celesti2010improving, malet2010resource}. In \cite{celesti2010improving} the so-called Composed Image Cloning (CIC) methodology designed for migration of VMs across federated clouds is introduced. This work aims to reduce the needed bandwidth and migration time by setting up a new virtual machine in the destination cloud and transferring only user data instead of relocating the whole VM disc image. 

In another work \cite{malet2010resource} the placement of applications is dynamically adjusted across distributed data centers according to the location of the currently highest request rate. Agarwal et al.~\cite{agarwal2010volley} provide a framework to minimize inter-data center traffic and user perceived latency by analyzing log files and thus determine the best placement and migration strategy. 



\section{Analysis}

\subsection{Relevance of topics}

\subsection{Similar approaches existing in literature}

\subsection{Implications and conclusion}


\section{Comparison and summary of existing approaches}

\subsection{Comparison by topics}

\subsection{Potential benefits of this work}

\subsection{Summary of investigation}


