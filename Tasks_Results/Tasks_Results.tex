
\documentclass[a4paper]{article}

\usepackage[english]{babel}
\usepackage[utf8]{inputenc}
\usepackage{csquotes}
\usepackage{amsmath}
\usepackage{graphicx}
\usepackage{hyperref}

% Load in biblatex
% To use a different bibliography style, just change "numeric" to
% your preferred style (mla for MLA style, alphabetic for Author-Year
% style, etc.) There are a lot of options; check the BibLaTeX documentation.
\usepackage[backend=bibtex8,style=numeric]{biblatex}
%% Select the bibliography file
%\addbibresource{sources.bib}


\title{Diploma Thesis\\Tasks Results}
\author{Andreas Egger \\0626885}
\date{Technical University of Vienna}


\begin{document}
\maketitle

\newpage

\tableofcontents

\newpage

\section{Outlining simulation}
\hfill\date{Week 07.04. to 13.04.}

\subsection{Defining simulation conditions}

\begin{itemize}

\item Defining countries and locations of data centers

There will be seven data centers located in seven different countries, each belonging to a specific energy market. 

Below is an outline of countries and cities of data center locations grouped by the corresponding energy markets. 

	\subitem European Energy Exchange (EEX) 
	
		\subsubitem - Germany: Berlin
		
		\subsubitem - Switzerland: Zurich
		
		\subsubitem - France: Paris
		
	\subitem Nord Pool Spot Market 
	
		\subsubitem - Finland: Helsinki
		
	\subitem APX Power Spot Market
	
		\subsubitem - England: London
		
	\subitem ISO New England
	
		\subsubitem - Maine: Portland
		
		\subsubitem - Massachusetts: Boston

\item Creating a visual outline of the geographical structure

A basic outline of the simulation scenario is shown in Figure~\ref{fig:simulation-outline}. 

The figure depicts a private cloud of a mid-sized company comprising several data centers distributed across different countries. 
%Basically there are seven data centers distributed across several countries. 
The dashed lined boxes group the data centers to the respective power market they belong to. The data centers are interconnected in a way that data may be transferred between any two data centers. 

Since data centers are distributed across Europe and the east coast of North America, interesting constellations may arise in terms of time and location based energy pricing and temperature levels, which may influence the amount of cooling needed which in turn affects the amount of power consumption needed for a specific data center, respectively. 


\begin{figure}[htbp]
	\centering
		\includegraphics[width=1.00\textwidth]{images/simulation_outline.png}
	\caption{Visual outline of the simulation scenario}
	\label{fig:simulation-outline}
\end{figure}




\item Defining parameters of the simulation

\emph{To be defined:}

	\subitem Data Center
	
		\subsubitem - Number of servers
		
		\subsubitem - Type of servers (CPU power, power draw)
		
		\subsubitem - Size of data center
		
		\subsubitem - Total power costs
	
	\subitem Cooling infrastructure
	
		\subsubitem - Number of Fans and/or Chillers per data center
		
		\subsubitem - Speed, trigger points related to outside temperature
		
		\subsubitem - Consider startup time
	
	\subitem Type of workload
		
		\subsubitem - Long running, data intensive workload
		
		\subsubitem - Short running requests
		
		\subsubitem - Simulating CPU usage behaviour
		
	\subitem Workload Generator (Load Balancer)
			
		\subsubitem - Responsible for distributing request to data centers
		
		\subsubitem - Decides on type of request
		
		\subsubitem - Current / future conditions (energy price, temperature levels) 
		
	\subitem Forecasting
	
		\subsubitem - short term (one day to a few days ahead)
		
		\subsubitem - specify methods in detail
		
	\subitem SLAs
	
		\subsubitem - allowable delay of tasks in seconds
		
		\subsubitem - minimum CPU power overhead for tasks in cycles

	\subitem Migration / Overhead costs
	
		\subsubitem - migrating VMs between data centers
		
		\subsubitem - Cost depending on distance, amount of data
		
	\subitem Minimum number of VMs to migrate
	
		\subsubitem - depending on migration costs and energy prices
			
	\subitem Power consumption behavior
	
		\subsubitem - Idle Servers
		
		\subsubitem - Partially utilized Servers (max. 1, negligible)
		
		\subsubitem - Fully utilized Servers
	
	\subitem Different kinds of energy sources
	
		\subsubitem - Renewables, Atom, Fossils
		
		\subsubitem - Possibly prioritizing green IT over price

	\subitem Definition On/Off Peak Prices
	
		\subsubitem - depending on respective power market
		
		\subsubitem - considering relevant time zones
		
	\subitem Different Timezones (By countries)

		\subsubitem - Maine, Massachusetts (-5) 

		\subsubitem - UK (UTC) 

		\subsubitem - Germany, France, Switzerland (+1) 

		\subsubitem - Finland(+2)
		
	
		

\item Collecting energy price data from the various energy markets

	\subitem European Energy Exchange
	
	EEX sets the energy prices for the countries in middle Europe and is an important platform for electricity price trading.  
	Data is available for the areas Switzerland, Germany/Austria and France, each describing a different part of the power market. It can be ordered and downloaded from an ftp server.
	
	website: \url{http://www.eex.com/}
	
	\subitem Nord Pool Spot Market
	
	The Nord Pool Spot Market is the largest power spot market in the world, with participating countries Norway, Denmark, Sweden, Finland, Estonia, Latvia and Lithuania. It contains three spot markets, Elspot, Elbas and N2EX  whereby Elspot is the most common one. It features different energy data attributes like Flow, Prices, Volumes and Capacities. It has a wide variaty of data available for the different countries. 
	
	website: \url{http://www.nordpoolspot.com/}
	
	\subitem APX Power Spot Exchange
	
The APX Power Spot market provides energy price data from the UK and the Netherlands. It comprises daily results and historical market data. Prices are given in Pounds. 

	website: \url{http://www.apxgroup.com/}

	\subitem ISO New England
	
	ISO New England contains data comprising the countries of the New England association, namely Connecticut, New Hampshire, Maine, Massachusetts, Rhode Island und Vermont. Two countries have been selected, Massachusetts and Maine, for which data is readily available. 
	
	website: \url{http://www.iso-ne.com/}

\item Examining data formats and ways to process them

	\subitem European Energy Exchange
	
	EEX offers infoproduct packages to get historical intraday spot prices and power future contracts (derivatives) for the years 2000 and 2002 until today, depending on the offered data format. 
	It provides its data in csv, xml and xls, partly containing different information. It is separated by year and country or market area, respectively. 
	

	\subitem Nord Pool Spot Market
	
	For the Nord Pool Spot Power market there are a number of data sets publicly available, however only from 2012 onwards. 
	The markets offer information about flow, price, volume and capacities in the given year. The Elspot market provides hourly, daily and weekly data, again freely available from 2012 onwards in files containing different nordic currencies.  

	\subitem APX Power Spot Exchange
	
	At APX Power Exchange just like at EEX data is provided via an ftp server. In order to get access one has to subscribe by email, responses are now on their way. 
	
	\subitem ISO New England
	
	At the ISO New England Power Spot market one can choose different types of historical energy price data, namely five-minute data, hourly data, daily data and monthly data. In addition there is data about short and long term outages available. 
	Data is given both in csv and xls format, for easy processing. 

\end{itemize}



\vspace{1em}

\hfill\date{Week 16, from 14.04. to 20.04.}

\section{Forecasting}

\subsection{Experiment on forecasting methods}

\begin{itemize}

\item Examine the forecasting methods in relation to the actual energy price data

\item Finding metrics for accuracy of the methods

\item Setting up a test environment for testing forecasts

\item Test forecasting for different time ranges (day ahead, few days)

\item Compare the accuracy of forecasts for different models

\end{itemize}


\vspace{1em}

\hfill\date{Week 17, from 21.04. to 27.04.}

\section{Energy Data and Forecasting}

\subsection{Experiment on forecasting methods (Re-work last week's tasks)}

\begin{itemize}

\item Examine the forecasting methods in relation to the actual energy price data

Neural Networks and Regression Trees (built-in Matlab functions) are candidates for 
electricity price forecasting. Based on calculated accuracy (see next thread) 
the most accurate method should be used or a combination of the best methods, respectively. 

\item Finding metrics for accuracy of the methods

MAE (Mean absolute error) and MAPE (mean absolute percent error) may be used to 
compare estimated to actual data. 

\item Setting up a test environment for testing forecasts

\subitem \emph{Collecting information about Matlab}

A great introduction for integrating energy price forecasts into MATLAB can be found on the MathWorks website\footnote{http://www.mathworks.de/discovery/load-forecasting.html}. 

A webinar on Electricity Load and Price Forecasting is also available\footnote{http://www.mathworks.de/videos/electricity-load-and-price-forecasting-with-matlab-81765.html}.

Another introduction to short term load forecasting is given in a youtube video, together with the freely available project\footnote{http://www.youtube.com/watch?v=0mareoVZyVg}.


\item Find a way to feed energy price data to Matlab forecasts

Import possible from .xls files via Matlab native load function (xlsread). 

Before consolidation / reformatting of the xls File might be necessary. 

Similar functions for reading xml and csv data. 

\item Test forecasting for different time ranges (day ahead, few days)

Day ahead forecasting is shown by example, where price data over one year 
is used to train models to predict the next single day. 

In order to forecast several days parameters have to be adjusted and possibly 
more historical data is needed for more accurate forecasts. 

Still under evaluation. 

\end{itemize}

\pagebreak
\vspace{1em}

\hfill\date{Week 18, from 28.04. to 04.05.}

\section{Definitions and documentation}

\subsection{Refining the tasks and direction of the work}

\begin{itemize}

\item Updating the expose regarding new insights

Rearrange phrases such that it makes sense regarding the simulation framework 
and VM schedulings. Adding relevant papers to the literature and provide insights on 
how they relate to the rest of the work. 

\item Decide which features to include

Omit cooling infrastructure and parameters since it is out of focus of this work. 

Instead provide in depth details about inter data center VM scheduling to optimize
energy consumption and therefore energy costs as well. 

\item Getting a clearer picture of which kind of data is available from which power market

Generally of interest are hourly data from the Spot markets, which provide the actual day ahead 
electricity price for the area associated with the power market. 

\subitem EEX

Historical hourly data from the spot market is available for students at a small charge, for 
the market areas germany, france and switzerland from the year 2000 until today. 

Data for Germany and Switzerland may be accessed as xls files, data from France is available in csv format. 

\subitem Nord Pool Spot

The Nord Pool Spot market provides free hourly data from 2012 until today, which might be just enough for 
meaningful short term forecasting. Data is available as xls files. 

\subitem APX UK Spot

Still no reply from the UK Power Spot market, in case it doesn't work it will be replaced by data from Sweden. 

\subitem ISO New England

Hourly data is available from the Spot market from the year 2003 until today. Thus sufficient amount of data
is available for short term forecasting measures. 

Day ahead and real time prices are available as well as dry bulb and dew point data. 

Data is provided as xls files. 

\item Further process forecasting methods in Matlab

Consolidating / importing excel data into Matlab. 

Understanding sample code and built-in models for machine learning and forecasting. 

\item Getting the connection between energy efficiency and costs

Being more energy efficient ultimately reduces energy costs as well if the overall 
energy consumption is decreased. This is the case if resources are consolidated onto
fewer servers such that idle servers may be shut off. Since there is more energy saved
when shutting down a server than is increased by an increase of load due to consolidation
there is less energy consumed in general. 

This makes up for less energy usage and therefore resulting in fewer energy related costs. 

\end{itemize}


\end{document}