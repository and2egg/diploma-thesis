%% Based on a TeXnicCenter-Template, which was
%% created by Christoph B�rensen
%% and slightly modified by Tino Weinkauf.
%%%%%%%%%%%%%%%%%%%%%%%%%%%%%%%%%%%%%%%%%%%%%%%%%%%%%%%%%%%%%

\documentclass[a4paper,12pt]{scrartcl} %This is a special class provided by the KOMA script, which does a lot of adjustments to adapt the standard LaTeX classes to european habits, change to [a4paper,12pt,twoside] for doublesided layout


%########################### Preferences #################################


% ********* Font definiton ************
\usepackage{t1enc} % as usual
\usepackage[latin1]{inputenc} % as usual
\usepackage{times}		
%\usepackage{mathptmx}  	%mathematical fonts for use with times, I encountered some problems using this package togather with pdftex, which I was not able to resolve

% ********* Graphics definition *******
\usepackage[pdftex]{graphicx} % required to import graphic files
\usepackage{color} %allows to mark some entries in the tables with color
\usepackage{eso-pic} % these two are required to add the little picture on top of every page
\usepackage{everyshi} % these two are required to add the little picture on top of every page
\renewcommand{\floatpagefraction}{0.7} %default:0.5 allows two big pictures on one page

%********** Enybeling Hyperlinks *******
\usepackage[pdfborder=000,pdftex=true]{hyperref}% this enables jumping from a reference and table of content in the pdf file to its target

% ********* Table layout **************
\usepackage{booktabs}	  	%design of table, has an excellent documentation
%\usepackage{lscape}			%use this if you want to rotate the table together with the lines around the table

% ********* Caption Layout ************
\usepackage{ccaption} % allows special formating of the captions
\captionnamefont{\bf\footnotesize\sffamily} % defines the font of the caption name (e.g. Figure: or Table:)
\captiontitlefont{\footnotesize\sffamily} % defines the font of the caption text (same as above, but not bold)
\setlength{\abovecaptionskip}{0mm} %lowers the distace of captions to the figure


% ********* Header and Footer **********
% This is something to play with forever. I use here the advanced settings of the KOMA script

\usepackage{scrpage2} %header and footer using the options for the KOMA script
\renewcommand{\headfont}{\footnotesize\sffamily} % font for the header
\renewcommand{\pnumfont}{\footnotesize\sffamily} % font for the pagenumbers

%the following lines define the pagestyle for the main document
\defpagestyle{cb}{%
(\textwidth,0pt)% sets the border line above the header
{\pagemark\hfill\headmark\hfill}% doublesided, left page
{\hfill\headmark\hfill\pagemark}% doublesided, right page
{\hfill\headmark\hfill\pagemark}%  onesided
(\textwidth,1pt)}% sets the border line below the header
%
{(\textwidth,1pt)% sets the border line above the footer
{{\it Technical University of Vienna}\hfill Andreas Egger}% doublesided, left page
{Andreas Egger\hfill{\it Technical University of Vienna}}% doublesided, right page
{Andreas Egger\hfill{\it Technical University of Vienna}} % one sided printing
(\textwidth,0pt)% sets the border line below the footer
}

%this defines the page style for the first pages: all empty
\renewpagestyle{plain}%
	{(\textwidth,0pt)%
		{\hfill}{\hfill}{\hfill}%
	(\textwidth,0pt)}%
	{(\textwidth,0pt)%	
		{\hfill}{\hfill}{\hfill}%
	(\textwidth,0pt)}

%********** Footnotes **********
\renewcommand{\footnoterule}{\rule{5cm}{0.2mm} \vspace{0.3cm}} %increases the distance of footnotes from the text
\deffootnote[1em]{1em}{1em}{\textsuperscript{\normalfont\thefootnotemark}} %some moe formattion on footnotes

%################ End Preferences, Begin Document #####################

\pagestyle{plain} % on headers or footers on the first page

\begin{document}

\begin{center}

\begin{figure}[th]
    \centering
		%\includegraphics[width=10cm]{logo.jpg}
	\label{fig:logo}
\end{figure}

\vspace{2cm}

% There might be better solutions for the title page, giving all distances and sizes manually was simply the easiest solution

{\Huge\bf\sf Notes }

\vspace{.5cm}

{\Huge\bf\sf for }

\vspace{.5cm}

{\Huge\bf\sf Diploma Thesis}

\vspace{2cm}

{\Large\bf\sf Andreas Egger}%as this is an english text I didn't load the german package, this would ease the use of special characters

\vspace{.5cm}

{\Large\bf\sf 0626885}

\vspace{2cm}

{\Large\bf\sf \today} %adds the current date

\vspace{\fill}

egger.andreas.1@gmail.com

\end{center}
\newpage

%%The following loads the picture on top of every page, the numbers in \put() define the position on the page:
%\AddToShipoutPicture{\setlength\unitlength{0.1mm}\put(604,2522){\includegraphics[width=1.5cm]{logo.jpg}}}

\pagestyle{cb} % now we want to have headers and footers

\tableofcontents

\newpage

\section{Expose}

\subsection{VM Migrations and Forecasting}

{\hfill\today}
\vspace{0.4cm}

Forecasting can only be made for individual workloads on different servers / data centers. 
But are forecasts of 5 to 10 days even possible when migration of virtual machines is included? 
Suppose a job runs on multiple VMs, and it is migrated in some point in time. Considering all servers and data centers
it is not possible to predict each migration, as it soon gets too complex. 
So the purpose of forecasts can only be a vague estimate of what might be advantageous in the future? 


\emph{BUT} of course in terms of energy prices forecasts perfectly make sense in that you can predict the price for
certain areas and data centers. Then you can check if the prices will significantly differ for a longer period of time 
in the future. It makes sense also for short term considerations. 

For example, there is a situation where a migration seems beneficial in that moment, but looking ahead one can see that
migrating wouldn't be worth it and it is better to do nothing. 


\subsection{Workloads and test data}

Open Questions:

\begin{itemize}

\item How should workloads be generated?
	\subitem -> workloads are CPU traces of programs, probably distributed across several CPUs (CPU power load)
\item Which programs should be used? 
	\subitem Matlab
	\subitem R
	\subitem Mathcad
	\subitem \ldots
\item Which kind of distribution should they exhibit (time series)?
\item Is there a way to get workload data / time series data from real data centers?

\end{itemize}

\subsection{Workload traces}

A workload trace is common for all data centers, since they are administrated from one central point. 
From there the workload is distributed across the data centers, already taking into account electricity cost. 
When a new request arrives, it is checked whether the application can be served at any of the data centers. 
If not, the start is delayed until resources are available. Otherwise it is assigned to the presumably cheapest
data center. If later the prices change load will be migrated from more expensive to cheaper data centers. 


\section{Simulation}

\subsection{Data Center Locations}

PHELIX: Physical Electricity Index: Germany, Austria

ELIX: Electricity Index: ?

SWISSIX: Swiss Index: Switzerland

FRENCH FINANCIAL POWER FUTURES: France



\end{document}