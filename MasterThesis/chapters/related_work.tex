

\section{Literature studies}

\subsection{Electricity markets and pricing}

In wholesale energy markets different pricing and bidding models can be used. 

Currently the two most common price evaluation strategies are day-ahead and real-time pricing strategies. (references?)

In \cite{tierney2008uniform} two different bidding strategies are discussed, uniform pricing and pay-as-bid auctions. In the uniform pricing model the market clearing price is determined by collecting the marginal prices from all suppliers and taking the maximum price from this collection. Conversely, in pay-as-bid auctions a supplier gets paid based on its actual bid. 
The second approach may seem beneficial from the customer's point of view since suppliers may set individual prices which enables competition within the market. 
However studies show that in this pricing scheme suppliers set their prices at the maximum possible level to be comparable to other suppliers and keep their customers. On the other hand the uniform pricing model provides a uniform clearing price which is valid for all participants in the market and customers may trust that suppliers set prices to just satisfy their needs. 

\subsection{Scheduling optimized for dynamic electricity prices}

Active research topics include electricity cost reduction in data centers \cite{guler2013cutting, le2011reducing}. A comparison of different approaches considering spatial and temporal energy price changes and the contribution of the outside temperature to the resulting cooling effort and expense is outlined in \cite{guler2013cutting}. In \cite{le2011reducing} a set of homogenous data centers is simulated where the focus lies on the effects of changing workloads and migrations on the cooling infrastructure and on intelligent scheduling to reduce energy consumption and energy costs, respectively. 

The authors in \cite{lucanin2013take} propose a green cloud approach using real time electricity prices where the duration and time of price peaks are estimated and VMs are paused during these times to reduce energy costs and save energy. Customers may choose between ``green'' and normal instances, taking into account some loss of availability for a reduced price. 

\subsection{Time series forecasting}

Time series analysis is an active research topic as it is important to find efficient algorithms that scale on very large amounts of data. A survey on machine learning models in time series forecasting is presented in \cite{ahmed2010empirical}, where the suitability of the models concerning forecasting is evaluated. Notably, some models did considerably better than others, therefore these models are good candidates for applications in time series forecasting. 

In \cite{lin2011pattern} a comprehensive benchmark on multivariate time series is described using similarity and distance measurement functions and pattern recognition techniques. In this work unsupervised / semi-supervised machine learning methods are investigated and a new similarity measure is introduced which exhibits a very high hit rate. 

\subsection{Virtual machine migration in cloud environments}

There are different proposed approaches related to virtual machine migration in distributed cloud environments \cite{celesti2010improving, malet2010resource}. In \cite{celesti2010improving} the so-called Composed Image Cloning (CIC) methodology designed for migration of VMs across federated clouds is introduced. This work aims to reduce the needed bandwidth and migration time by setting up a new virtual machine in the destination cloud and transferring only user data instead of relocating the whole VM disc image. 

In another work \cite{malet2010resource} the placement of applications is dynamically adjusted across distributed data centers according to the location of the currently highest request rate. Agarwal et al.~\cite{agarwal2010volley} provide a framework to minimize inter-data center traffic and user perceived latency by analyzing log files and thus determine the best placement and migration strategy. 



\section{Analysis}


\section{Comparison and summary of existing approaches}

