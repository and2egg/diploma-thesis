

\section{Energy markets}

Energy markets have emerged steadily from the mid-eighties (Research history!!). 






\subsection{Types of energy markets}

Two main models exist for the exchange and types of trading of energy prices in power markets which are bilateral trading and electricity pooling \cite{onaiwu2009does,hogan1997reshaping,barroso2005classification}.

In bilateral trading utility operators (producers) and energy consumers establish bilateral contracts to determine the terms and conditions applied for trading energy \cite{onaiwu2009does}. Generators or producers may also buy electricity from other suppliers in case their own electricity generation does not fit current demand. The exact amount, price and time when tradings take place are negotiated on the contract. Since electricity demand may vary from the negotiated terms the resulting imbalances have to be taken care of by the system operator. 

Electricity pooling provides a way for utility operators to offer a certain amount of energy at a set price. Each generator places a bid containing the quantity and expected price \cite{barroso2005classification}. In return consumers place bids on how much they are willing to pay for a given amount of energy. The intersection of the aggregated demand and supply curves defines the energy price for that hour. 

Still customers, brokers and aggregators have the choice to either establish long term contracts of electricity or rely on the short-term market \cite{hogan1997reshaping}.


\subsubsection{Day ahead vs real time markets}

Both day ahead and real time markets exist within the scope of a deregulated wholesale power market. As discussed above, electricity pooling is a flexible way of determining energy prices for a short period of time in the future \cite{hogan1997reshaping}. Prices are set based on current demand and supply and are fixed for a particular point in time. This price is then valid for all market participants during that time period. 

In day ahead markets bids are placed for each consecutive hour of the following day. Thus prices are determined on a very short timescale and may exhibit significant volatility. Prices may change substantially from one hour to the next, up to a factor of ten \cite{huisman2007hourly}. Therefore market operators need to apply profound risk management techniques to alleviate such price spikes. 
%it is hard to predict

The real time market on the other hand is used to compensate demand variations during the day which may result in additional changes in energy prices. 
Due to this characteristic the real time market prices exhibit an even greater volatility than those at the day-ahead market whereby the latter is generally used for reference. 

TODO Compare real time (5 min?) data with hourly day ahead data (graph). Evaluate volatility and maybe also forecasting behavior (this better suits under forecasting menu). 


\subsection{Summary of selected energy markets}



\subsubsection{Nord Pool Spot}


The NordPoolSpot market is the most mature power market to date and has been the first energy market that was established in Europe [c]. 

The NordPoolSpot power market consists of four different markets. [c BEGIN]

\begin{itemize}[-]

\item Financial market (Nasdaq OMX)
Used for managing risks. Cash settled futures,
forwards and options. Contracts can be made for
up to six years. The system price is used as
reference price.

\item Day-ahead market (Nord Pool Spot)
Day-ahead auction of power for delivery the next
day. Nord Pool Spot calculates power prices
based on supply and demand for every hour the
following day.

\item Intraday market (Nord Pool Spot)
Continuous trading up to 30 minutes before
delivery to adjust power production or
consumption plans.

\item Balancing market (TSOs)
Operated by the respective transmission system
operators. Final adjustments are made to ensure
the correct frequency in the grid and security of
supply

\end{itemize}

\emph{The future is Europe}

The future of power trading lies in the
creation of a single, integrated
European market and an integrated
platform.
This will be the most advanced power
market yet, offering efficiency in:

\begin{itemize}
\item Membership
\item Management of collateral
\item Cost
\end{itemize}

[c END] - Nord Pool Spot - Europe's leading power market \url{http://www.nordpoolspot.com/globalassets/download-center/annual-report/nord-pool-spot_europes-leading-power-markets.pdf}




Nord Pool Spot - Day ahead market

[c BEGIN]

The day-ahead market, Elspot , is the main arena for trading power in the Nordic and Baltic region. Here, contracts are made between seller and buyer for the delivery of power the following day, the price is set and the trade is agreed.

Today there are around 360 buyers and sellers (called members) on Elspot. Most of them trade every day, placing a total of around 2000 orders for power contracts on a daily basis.

\emph{Driven by planning}

Daily trading is driven by the members’ planning. A buyer, typically a utility, needs to assess how much energy (‘volume’) it will need to meet demand the following day, and how much it is willing to pay for this volume, hour by hour. The seller, for example the owner of a hydroelectric power plant, needs to decide how much he can deliver and at what price, hour by hour. These needs are reflected through orders entered by buyers and sellers into the Elspot trading system.

\emph{Setting the price and closing the deal}

12:00 CET is the deadline for submitting bids for power which will be delivered the following day. Elspot feeds the information into a specialist computer system which calculates the price, based on an advanced algorithm. Put simply, the price is set where the curves for sell price and buy price meet.

\emph{Supply and demand}

Hourly prices are typically announced to the market at 12:42 CET or later. Once the market prices have been calculated, trades are settled. From 00:00 CET the next day, power contracts are physically delivered (meaning that the power is provided to the buyer) hour for hour according to the contracts agreed.

\emph{The cost of transmission constraints}

While supply and demand are the key factors determining the hourly market prices, transmission capacity also plays a role. Bottlenecks can occur where power connections are linked to each other, if large volumes need to be transmitted to meet demand. To relieve this congestion, different area prices are introduced. In other words, when transmission capacity gets constrained, the price is raised to reduce demand in the areas affected.

[c END] - Nord pool Spot Elspot Market \url{http://www.nordpoolspot.com/How-does-it-work/Day-ahead-market-Elspot-/}



Nord Pool Spot - Intraday market

[c BEGIN]

Elbas  is an intraday market for trading power operated by Nord Pool Spot. Covering the Nordic and Baltic region as well as Germany and recently extended to include the UK, Elbas supplements Elspot and helps secure the necessary balance between supply and demand in the power market for Northern Europe.

The majority of the volume handled by Nord Pool Spot is traded on the day-ahead market. For the most part, the balance between supply and demand is secured here. However, incidents may take place between the closing of Elspot at noon CET and delivery the next day. A nuclear power plant may stop operating in Sweden, or strong winds may cause higher power generation than planned at wind turbine plants in Germany. At Elbas, buyers and sellers can trade volumes close to real time to bring the market back in balance.

Trading close to real time

At 14:00 CET, capacities available for Elbas trading are published. Elbas is a continuous market, and trading takes place every day around the clock until one hour before delivery. Prices are set based on a first-come, first-served principle, where best prices come first – highest buy price and lowest sell price.

Increasingly important

The intraday market is becoming increasingly important as more wind power enters the grid. Wind power is unpredictable by nature, and imbalances between day-ahead contracts and produced volume often need to be offset. Elbas will play a key role in the development of intraday power trading in Europe. Future prospects indicate exponential growth, reaching 1.900 GW installed wind capacity worldwide in 2020 (Source: World Wind Energy Association). This type of market can be a key enabler to increase the share of renewable energy in the energy mix. 

[c END] - Nord pool Spot Intraday Market \url{http://www.nordpoolspot.com/How-does-it-work/Intraday-market/}




\subsubsection{ISO New England}

The market of ISO New England is located at \url{http://www.iso-ne.com/}. It offers day ahead as well as real time energy prices where day ahead prices are available at an hourly interval while (preliminary) real time prices are available both at hourly and five minute intervals. 

Current locational marginal prices (LMPs) can be downloaded from a base url\footnote{\url{http://www.iso-ne.com/static-transform/csv/histRpts/da-lmp/}} and a specialized suffix depending on the file and date (e.g.~WW\_DALMP\_ISO\_20150414.csv). In this case, files are provided in csv format for better machine processing. 

Example: 
\begin{verbatim}
	H,"Date","Hour Ending","Location ID","Location Name","Location Type","Locational Marginal Price",
	"Energy Component","Congestion Component","Marginal Loss Component"
	D,"04/14/2015","02","4006",".Z.SEMASS","LOAD ZONE",14.82,14.85,0.01,-0.04
	D,"04/14/2015","02","4007",".Z.WCMASS","LOAD ZONE",14.94,14.85,0.01,0.08
	D,"04/14/2015","02","4008",".Z.NEMASSBOST","LOAD ZONE",14.84,14.85,0.01,-0.02
\end{verbatim}

A summary of all data that is available is provided for day ahead as well as real time prices\footnote{\url{http://www.iso-ne.com/isoexpress/web/reports/pricing/-/tree/zone-info}}. It is described as ``zonal information'' on the website to emphasize the locational character of the prices. 



\subsubsection{PJM}

See at \url{http://www.pjm.com/}. 


\subsubsection{OMIE - Spanish power market}

OMIE represents the wholesale energy market for the iberian region. \footnote{\url{http://www.omel.com/en/inicio}}

EDP (Energias de Portugal) is an energy company in cooperation with the wholesale electricity market for Spain, Portugal and is also responsible for delivery and transmission of energy to other countries. \footnote{\url{http://www.edp.pt/en/aedp/sectordeenergia/sistemaelectricoespanhol/Pages/SistElectES.aspx}}



\subsection{Terms and definitions}

\subsubsection{Definition of 'Clearing Price'}

The following is a definition of clearing price from Investopedia\footnote{\url{http://www.investopedia.com/terms/c/clearingprice.asp}}. 

\begin{quote}
The specified monetary value assigned to a security or asset. This price is determined by the bid and ask process of buyers and sellers interested in trading the security.

In any exchange, sellers prefer to part with their assets for the highest price possible while investors interested in buying the same asset desire the lowest purchase price possible. At some point, a mutually agreeable price is reached between buyers and sellers. It is at this point that economists say the market has "cleared" and transactions take place. Thus, the clearing price of an asset is the price at which it was most recently traded.
\end{quote}


\subsubsection{Locational marginal price}

TODO explain LMP (-> see ISO New England energy data, smd hourly!)

\subsubsection{System Load}

TODO

Some energy markets provide the total system load as measured by metering. It is used for day-ahead and long term forecasting purposes and in addition it serves as important indicator in reports. The system load for the ISO New England energy market is calculated by $System Load = generation - pumping + net\_interchange$


\subsubsection{Forward Capacity Market (FCM)} 

A forward capacity market is used to ensure having enough capacity for a specific amount of time into the future. A capacity commitment period (CCP) ensures that a certain amount of capacity will be available during that period. For example, the CCP at ISO-NE is set as one year ranging from June 1st until May 31st. 

The market has to ensure that the \emph{Installed Capacity Requirements (ICR)} for the corresponding region are met. These requirements are defined for each capacity commitment period and define the amount of capacity needed to meet estimated peak and reserve demands. Important measures to define the ICR are the local sourcing requirements (LSR) and maximum capacity limits (MCL) that define the constraints given by market participants. %(QUOTE)In addition the Hydro-Québec Interconnection Capability Credits (HQICCs), are a key input into the calculation of the ICR.

The actual procurement of resources for each capacity commitment period is determined by a \emph{Forward Capacity Auction (FCA)} which aims to meet the defined ICRs for the given period.
This auction is carried out three years ahead of the related CCP to ensure that enough resources will be available during that period. \emph{Reconfiguration Auctions (RA)} are then executed annually until the start of the CCP and continued monthly afterward. During an RA capacity resources may be amended to adapt to potential changes in capacity zones. 

At the FCM of ISO New England there is also the option of making a composite offer where different capacity resources may join their capacity offers (useful i.e.~in case of single season capacities) to result in a single resource offer at the market. 

Capacity resources <-> energy suppliers/providers?

See definition at the website of ISO NE\footnote{\url{http://www.iso-ne.com/markets-operations/markets/forward-capacity-market}}.

\subsubsection{Regulation market}

Definition by the ISO NE power market\footnote{\url{http://www.iso-ne.com/markets-operations/markets/regulation-market}}:

\begin{quote}
The Regulation Market is the mechanism for selecting and compensating market participants to provide regulation—the capability of specially equipped generators and other energy sources to increase or decrease output or consumption every four seconds. Participants allow their Automatic Generation Control (AGC) resources to be controlled by the ISO using automated signals to balance both second-by-second variations in demand and the system frequency, which must be kept constant. This market helps ensure that the ISO meets the North American Electric Reliability Corporation Real Power Balancing Control Performance Standard (BAL-001-0). Two regulation clearing prices are calculated: one for capacity and one for actual service mileage.
\end{quote}




\subsubsection{Two settlement system: Day ahead vs Balancing market}

A two settlement system has been defined for the PJM power market (PJM standing for Pennsylvania, Jersey, Maryland)\cite{lambert2001creating}. 

A two-settlement system in power markets is understood as a system comprising a day ahead as well as a real time (balancing) power market\cite{lambert2001creating}. 




\section{Forecasting}

\subsection{Introduction}

Since the occurrence of competitive energy markets forecasting of energy prices has been vital to utility operators. 


\subsection{Forecasting models}

TODO Discussion of suitable models for price series from energy markets


\subsubsection{Random walk}

A random walk model is best suited for series that exhibit major fluctuations on a short term and where no apparent trend can be recognized \cite{makridakisforecasting}(pg. 461). Thus this model can be described by an accumulation of a random error over time (equation \ref{eq:random_walk}). 

\begin{equation} \label{eq:random_walk}
Y_t = \sum e_t
\end{equation}

$Y_t$ denotes the value of the time series at time point $t$ whereas $e_t$ is a timeseries of random errors where the errors exhibit no correlation and are normally distributed \cite{makridakisforecasting}(pg. 461). 

In \cite{makridakisforecasting}(pg. 464) it is stated that the behavior of economic and business series in particular can be characterized by a random walk model since they show so-called cycles or random fluctuations around a possible trend. 

As it is impossible to accurately predict upcoming values or cycles of a random walk model by definition one of the most suitable prediction models is the ``naive forecast'' where the forecasted value of the upcoming timestamp is set equal to the last observed value (equation \ref{eq:naive_forecast}).

\begin{equation} \label{eq:naive_forecast}
\hat{Y}_{t+i} = Y_t
\end{equation}



\subsubsection{Forecasting benchmarks}



\subsubsection{ARIMA models}

\emph{ARIMA model generation}

Box-Jenkins Methodology



Neural network forecasting competition (2008): 

\url{http://www.neural-forecasting-competition.com/NN5/}


\subsection{Accuracy measures}


\subsection{Model selection}


\subsection{Energy price related forecasts}



\section{Cloud-based Simulator}


\subsection{Motivation}

\subsection{Structure of simulator}

\subsection{Functional requirements}

TODO Discuss impact of evaluation of bandwidth costs. In \cite{rao2010minimizing} (page 7) bandwidth costs are deliberately neglected, but the option should be discussed at least. 

\subsection{Non-functional requirements}

\subsection{Incorporating forecast models}

As already discussed in (forecast chapter? TODO) forecasting is an integral component of the simulation as it should support in decision making when assigning workload and performing migrations. Since forecast errors can have a negative impact on workload allocation which may result in non-optimal workload distributions \cite{de2013study} it is of paramount importance that the forecast models are trained thoroughly and forecast errors are kept at a minimum. In order to reduce the impact of forecast errors in the simulation forecasting is done several hours into the future whereby the results are averaged to obtain a broader estimate of energy prices of the near future. 


\subsection{Simulation runs}



\section{Cost models}

Cost models are required to map the servers' energy consumption to costs depending on current energy prices. Different cost models exist exhibiting higher or lower accuracy and differ in their type of calculation. 

In this thesis several assumptions have been made: 

\begin{itemize}

\item \textbf{idle power.} A server has a defined idle power (e.g.~100 Watts (W)). This is the power that is drawn when the server is in idle mode (no tasks are running). 

\item \textbf{peak power.} A server has a defined peak power (e.g.~200 Watts (W)). This is the power that is drawn when the server runs on maximum load (maximum number of tasks running). 

\item \textbf{utilization.} Each server has a utilization between 0 and 100 percent. A utilization of zero percent means the server is idle, in contrast a utilization of hundred percent means the server is at peak load. 

\item \textbf{power consumption.} The power consumption is defined as the effective power consumed by a server during a certain period of time (e.g.~one hour). The result is a measure of power consumed per time (e.g.~150 Watt hours (Wh) for a server running on 50\% utilization for one hour)\footnote{\url{http://whatis.techtarget.com/definition/kilowatt-hour-kWh}}. %Energy can also be measured in Joule (J) where one Watt (W) equals 1 Joule per second (J/s)\footnote{\url{http://en.wikipedia.org/wiki/Kilowatt_hour}}. Thus one kilowatt hour equals $3.6 * 10^6$ kilojoules. 

\item \textbf{energy price.} The energy price is taken from the energy market that the data center is connected to. These prices typically change every hour which is therefore the relevant time interval for all further operations. 

\end{itemize}

From these assumptions a simple cost model can be derived. Since power is measured per hour in Watts and prices are given per Kilowatt Hours (kWh) the power consumed by servers can be directly mapped to energy costs. This model is simplified in that it does not consider frequency scaling or other power measures taken for a server (suspend, sleep mode). Thus it serves as a base for comparing different scenarios and it is also useful to get a first impression on the overall power consumption and costs involved in a given scenario. 

A cost model is essential for evaluating the usefulness of the approach presented in this thesis as the ultimate goal is to reduce energy related costs of geographically distributed and connected data centers based on changing energy prices. 





\section{power consumption}

According to a study from the Borderstep Institute of Germany\footnote{\url{http://www.bitkom.org/files/documents/Kurzstudie__Borderstep_l_Rechenzentren.pdf}} the power consumption of servers in Germany alone amounted to 9.7 terawatt hours in 2011 which equalled to about 1.8\% of Germany's total power consumption. From 2008 there has already been a reduction in energy consumption by 1.4 terawatt hours but another 2.3 TWh would have been possible if a "`Green IT"' Scenario would have been applied. Green IT means that only latest technology should be used combined with energy aware hardware to reduce power consumption of servers. 

Another approach to reduce energy consumption is to reduce cooling costs, since they represent a significant part of the overall energy expenses[c]. This relates to the power usage effectiveness (PUE) which indicates the relation of energy expense caused by IT equipment to the overall energy expenses of a data center. It is commonly used as an indicator to show the level of energy efficiency of a data center. 
