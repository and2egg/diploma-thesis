

\section{Motivation}

In times of cloud computing where computer resources are expected to be readily available from anywhere \cite{buyya2009cloud} and with an ever growing global network of interconnected devices cloud providers need to cope with increasing energy consumption in data centers as requests need to be handled at a rapid pace. 
As energy consumption is directly mapped to energy costs, cloud providers aim to reduce power consumption and thus electricity costs within data centers. 

Studies show that electricity costs for cloud systems comprising several large scale data centers is in the range of tens of millions of dollars per year \cite{qureshi2009cutting}. Energy consumption, related energy costs and high energy prices may slow down or hinder advancements in cloud computing. Thus, cloud providers must ensure that delivering their services remains profitable. Lower energy costs increase the competitive ability of cloud providers and bigger investments are possible. Likewise customers may benefit from more affordable cloud services. 

Energy cost reductions can be addressed by utilizing one of two different approaches. One common approach is to increase energy efficiency in data centers to reduce energy consumption and therefore energy related costs as well [c]. This is done i.e.~by utilizing more energy efficient hardware, implementing an energy efficient cooling system, virtualization of servers or energy aware load distribution policies. It additionally aims at minimizing the Power Usage Effectiveness (PUE) of data centers which describes the ratio of the total amount of power consumed to the power consumed solely by IT equipment\footnote{http://www.itwissen.info/definition/lexikon/power-usage-effectivness-PUE.html} [r]. Despite the importance of increasing energy efficiency to reduce costs and environmental load this approach is limited as energy efficiency may not be improved beyond a critical point where investments in energy efficiency measures do not further reduce overall energy expenses [g]. 
%Thus the expenses of auxiliary systems should be reduced to a minimum to spend less on electricity. 
%Two common approaches address the reduction of energy costs in data centers. 

The second approach is directed at intelligent scheduling mechanisms based on dynamically changing energy prices. In wholesale power markets energy prices commonly change at an hourly basis with factors up to 10 from one hour to the next [c]. Electricity price behavior of power markets may differ significantly with possibly involved seasonality patterns where different time zones lead to substantial locational price differences. These characteristics of locational and temporal varying energy prices may be exploited by intelligent scheduling algorithms and cost aware resource management routines that place requests at facilities currently exhibiting low energy prices. In addition the current energy price time series at any location and time stamp is observed and resources may be migrated to data centers that currently operate below peak load and facing low energy prices. %where energy prices are currently cheaper. 

The focus of this thesis is to go beyond intelligent scheduling of resources %to fit the current distribution of energy prices 
by integrating sophisticated forecasting methods that predict price time series behaviors into the near future. This measure is expected to further optimize resource distribution across data centers by comparing and analyzing current and predicted energy price time series to find the best fit of resources in the near future which should result in the maximum possible energy cost reduction.

%The goal of this thesis is to reduce energy related costs of interconnected data centers with respect to changing energy prices in wholesale power markets. 
%The proposed approach is to optimize load placement and migration between data centers considering time and location based energy prices. 
%In addition to placing resources at locations currently facing low energy prices the resources or virtual machines are migrated to sites where low energy prices are \emph{expected} in the near future. 

%The scenario includes data centers located in different countries where different energy markets and prices apply. Thus constellations may arise where it is profitable to migrate a certain amount of resources from a data center facing high energy prices to a data center where energy prices are low at the current point in time or in the near future. 

\section{Problem Statement}

Energy prices are traded and change hourly within the energy markets, so investigations need to be made in order to make qualified statements about the estimated energy price at some point in time in the near future.
This is done by analyzing historical price data and building models for accurate energy price forecasting. Energy prices belonging to different energy markets are estimated for various time ranges into the future. Based upon the results it is decided how many resources may be migrated currently or in the future in order to obtain the maximum possible energy cost reduction. 
However, forecasting errors represented by the difference between estimated and real energy prices increase with the length of the estimation window which has to be taken into account when determining the length of the forecasting window. 

Since energy prices change dynamically a flexible approach of moving data center resources to more cost-effective locations is needed. This work is based on the scenario of a company running data centers distributed across several countries. Depending on the current and estimated energy price for a certain region resources may be relocated to other data centers located at more cost-effective regions.

The key technology to implement the relocation of data center resources is virtual machine (VM) migration \cite{nelson2009virtual}. All data and services are located on servers and belong to specific virtual machines within a data center. Since a server may host several VMs that operate completely independent from each other, they can be relocated (migrated) to other servers, possibly over networks without affecting VMs running on the same server. VM migration may be applied across large geographical distances (geo migrations) by leveraging connections via the internet. It is also possible to do live migrations on a large scale without noticeable service interruption \cite{celesti2010improving}. 

Various load policies exist to optimize the load factor and distribution of tasks within data centers \cite{buyya2010energy}. A workload manager has to make sure that there are still servers having enough resources and no service level agreement (SLA) is violated on the assignment of a new job. 
In contrast, this work aims to optimize workload distribution among geographically distributed data centers with a focus on the reduction of overall electricity costs. This is done by intelligent scheduling while utilizing forecasting methods and predicting energy prices within the near future. 
Furthermore, by considering the history of energy prices and resource load within a certain region it is estimated whether or not a migration to another data center makes sense regarding the current load and configuration. Therefore, forecasting methods in combination with machine learning techniques will be evaluated and applied to the scenario. 

\section{Aim of the work}

The goal of this thesis is to show ways for data center operators in charge of geographically distributed and interconnected data centers to save on their power bill by exploiting VM migrations and the variability of energy prices in wholesale power markets. This approach presumes that data center operators are charged directly by wholesale power markets, which can be a suitable option since power market integration can effectively increase the power usage efficiency (PUE) within a data center\footnote{Data Center Trends: Optimizing power markets \url{http://www.datacenterworld.com/fall2013/account/Uploader/uploader\_files/show/335/}}. Given the fact that data centers consume a huge amount of energy comparable to the energy consumption of small cities\cite{qureshi2009cutting} it becomes a viable option for them to be integrated into wholesale energy markets. By leveraging the technique of resource migration to data centers where cheaper energy prices apply at the current point in time significant savings in energy bills are possible while still maintaining a defined level of quality of service. 

Initially placing resources at data centers that currently exhibit cheap energy prices can reduce energy expenses significantly compared to a non power aware approach. Combined with energy price forecasts resources may be scheduled intelligently such that price changes within the near future can be detected and resources are placed with respect to the estimated prices. This approach may be evaluated against an ad-hoc approach where resources are assigned to data centers by only considering current energy prices and workload. 

A simulation scenario will be established where workload is placed on several geographically distributed and interconnected data centers and results are evaluated based on the given workload, energy prices and data center capacities. In addition, energy price forecasts will be integrated to predict significant changes in energy price levels at different energy markets. This should lead to a more cost efficient scheduling approach where workload is placed at data centers which provide best energy price conditions for current and/or future time spots. 

The simulation will be based on a well established cloud scheduler\footnote{\url{http://philharmonic.github.io/}} with set parameters to simulate a predefined scenario. The results for different scenarios and parameter settings (i.e.~with or without forecasts) will be compared and evaluated such that the possible benefits and drawbacks of each of these approaches can be examined. 
In addition to energy prices and workload this scheduler may take into account cooling costs related to ambient temperatures which can take up a significant part of data center expenses. 

A special focus will be laid on evaluating energy price forecasting methods to find the best fit for the given data and scenarios. For the simulation there will be a focus on short term price forecasting, as energy price levels change rapidly within energy markets and exhibit a very volatile nature. Therefore accurate price forecasting is only possible when applied to short term forecasts. 
Different forecasting methods will be compared by accuracy and error measures to determine the most suitable forecasting horizon and parameter settings for various models and data. It is expected that through good quality forecasting the results of the simulation in terms of energy costs can be improved significantly. This should help cloud providers to save on their energy bills and provide more efficient scheduling of cloud related data. 

\section{Methodological Approach}

The proposed scenario consists of a private cloud with data centers located in countries where participation in power markets is possible and energy price data is available. The geographical location and connections between data centers should be defined such that VM migrations may be done interchangeably. 

To maximize the predictive accuracy of price forecasting different parameters of forecasting methods should be evaluated and compared to determine the settings with the most accurate results in each case. The accuracy should be evaluated for various time ranges on both in-sample and out-of-sample forecasts and each location and time series separately. Due to the volatile nature of hourly real time energy prices there is a natural boundary of about 48 hours up to which forecasts can be made with reasonable accuracy. Models should be trained and compared based on forecast error measures on in-sample data (i.e.~MAPE) to determine the most accurate models. 

As a next step a model is to be defined for the simulation with parameters needed for further operation. For example, the amount of bandwidth between two data centers needs to be set and a method for calculating the cost of VM migration has to be defined, depending on current energy prices. Furthermore the minimum number of VMs should be set that may be migrated in order to compensate for migration costs. 

Regarding forecasting models there is evidence that some models are generally more suitable for energy price forecasting than others \cite{weron2008forecasting}\cite{weron2005forecasting}. For example, for averaged short term forecasts a common model to use is simple exponential smoothing (SES) which averages out any variations in the data and provides a good stationary estimate of future data values. In many cases this generates results with sufficient accuracy due to a strong irregular component (random fluctuations) in the data set. 

Another common model for price forecasting is the ARIMA model which may contain both an auto-regressive (AR) and moving-average (MA) part and can thus be applied to various data sets depending on the chosen parameters. Also, various forms of auto-regressive models exist which may provide a good fit for mid term forecasting. 
Therefore representatives from each of these models will be chosen where the focus lies on predicting real-time energy prices as they are collected on an hourly basis in contrast to the day-ahead spot energy prices which are provided for the next 24 hours simultaneously. 

\section{Structure of the work}

The structure of this thesis consists of five main parts. 

In the introduction a general outline is described comprising the purpose of the thesis, the problem statement and how these problems are solved within the thesis. Short insights are given into proposed methodologies and how they are expected to provide improvements to the outlined scenario. 

State of the art discusses relevant literature in the field with a comprehensive collection of related articles and journals. It is examined what kind of advancement in the different fields have been made in the past that relate to this thesis and how they can be used as a basis for further investigations. After a detailed analysis of the different approaches of solving problems in the respective field the approaches are compared and summarized to discuss what is still missing and can be improved upon or further investigated. 

Methodology comprises definitions and concepts used in further chapters and provides the basis for the discussion of the proposed solution. It investigates in detail theoretical models and concepts which were derived from extensive research in the field. It gives an overview of different energy markets, their inherent characteristics and related energy price time series behavior. The characteristics of electricity prices in wholesale power markets is discussed, as well as different pricing models applicable to these markets. State of the art data centers are investigated to get an idea of the possibilities and limitations of these systems. Finally relevant forecasting models and their theoretical background are discussed to better understand their applications within the thesis. 

The suggested solution or implementation shows the results and outcome of the research in previous chapters. It comprises the general idea and background for the proposed approach, a description of the technical architecture and structure of the implementation and possible interaction with other systems or components. The model selection process is explained in detail and how it contributes to the overall solution. This is crucial as good quality forecasting depends on the goodness of fit of the model in relation to the given data or time series. The empirical evaluation chapter describes the application of selected models to the framework and its behavior when assigned to real world data from different power markets. 
An important part of the evaluation is done on a simulator that mimics a cloud environment consisting of several distributed data centers. Considering changing energy prices, ambient temperatures, cooling efficiencies, scheduling of virtual machines combined with energy price forecasting it can provide comprehensive insights on possible cost reductions by applying forecasting in cloud environments. 

The discussion and critical reflection contains an overall discussion and investigation of the proposed solution and its possible benefits and limitations. It is compared to related work and how it fits into current research topics. Potential improvements to the proposed implementation are discussed with a reference of open issues. 
Finally the results are summarized and potential future advancements in achieving cost reductions in cloud environments are estimated. A future outlook concludes the work. 


